
% Default to the notebook output style

    


% Inherit from the specified cell style.




    
\documentclass[11pt]{article}

    
    
    \usepackage[T1]{fontenc}
    % Nicer default font (+ math font) than Computer Modern for most use cases
    \usepackage{mathpazo}

    % Basic figure setup, for now with no caption control since it's done
    % automatically by Pandoc (which extracts ![](path) syntax from Markdown).
    \usepackage{graphicx}
    % We will generate all images so they have a width \maxwidth. This means
    % that they will get their normal width if they fit onto the page, but
    % are scaled down if they would overflow the margins.
    \makeatletter
    \def\maxwidth{\ifdim\Gin@nat@width>\linewidth\linewidth
    \else\Gin@nat@width\fi}
    \makeatother
    \let\Oldincludegraphics\includegraphics
    % Set max figure width to be 80% of text width, for now hardcoded.
    \renewcommand{\includegraphics}[1]{\Oldincludegraphics[width=.8\maxwidth]{#1}}
    % Ensure that by default, figures have no caption (until we provide a
    % proper Figure object with a Caption API and a way to capture that
    % in the conversion process - todo).
    \usepackage{caption}
    \DeclareCaptionLabelFormat{nolabel}{}
    \captionsetup{labelformat=nolabel}

    \usepackage{adjustbox} % Used to constrain images to a maximum size 
    \usepackage{xcolor} % Allow colors to be defined
    \usepackage{enumerate} % Needed for markdown enumerations to work
    \usepackage{geometry} % Used to adjust the document margins
    \usepackage{amsmath} % Equations
    \usepackage{amssymb} % Equations
    \usepackage{textcomp} % defines textquotesingle
    % Hack from http://tex.stackexchange.com/a/47451/13684:
    \AtBeginDocument{%
        \def\PYZsq{\textquotesingle}% Upright quotes in Pygmentized code
    }
    \usepackage{upquote} % Upright quotes for verbatim code
    \usepackage{eurosym} % defines \euro
    \usepackage[mathletters]{ucs} % Extended unicode (utf-8) support
    \usepackage[utf8x]{inputenc} % Allow utf-8 characters in the tex document
    \usepackage{fancyvrb} % verbatim replacement that allows latex
    \usepackage{grffile} % extends the file name processing of package graphics 
                         % to support a larger range 
    % The hyperref package gives us a pdf with properly built
    % internal navigation ('pdf bookmarks' for the table of contents,
    % internal cross-reference links, web links for URLs, etc.)
    \usepackage{hyperref}
    \usepackage{longtable} % longtable support required by pandoc >1.10
    \usepackage{booktabs}  % table support for pandoc > 1.12.2
    \usepackage[inline]{enumitem} % IRkernel/repr support (it uses the enumerate* environment)
    \usepackage[normalem]{ulem} % ulem is needed to support strikethroughs (\sout)
                                % normalem makes italics be italics, not underlines
    

    
    
    % Colors for the hyperref package
    \definecolor{urlcolor}{rgb}{0,.145,.698}
    \definecolor{linkcolor}{rgb}{.71,0.21,0.01}
    \definecolor{citecolor}{rgb}{.12,.54,.11}

    % ANSI colors
    \definecolor{ansi-black}{HTML}{3E424D}
    \definecolor{ansi-black-intense}{HTML}{282C36}
    \definecolor{ansi-red}{HTML}{E75C58}
    \definecolor{ansi-red-intense}{HTML}{B22B31}
    \definecolor{ansi-green}{HTML}{00A250}
    \definecolor{ansi-green-intense}{HTML}{007427}
    \definecolor{ansi-yellow}{HTML}{DDB62B}
    \definecolor{ansi-yellow-intense}{HTML}{B27D12}
    \definecolor{ansi-blue}{HTML}{208FFB}
    \definecolor{ansi-blue-intense}{HTML}{0065CA}
    \definecolor{ansi-magenta}{HTML}{D160C4}
    \definecolor{ansi-magenta-intense}{HTML}{A03196}
    \definecolor{ansi-cyan}{HTML}{60C6C8}
    \definecolor{ansi-cyan-intense}{HTML}{258F8F}
    \definecolor{ansi-white}{HTML}{C5C1B4}
    \definecolor{ansi-white-intense}{HTML}{A1A6B2}

    % commands and environments needed by pandoc snippets
    % extracted from the output of `pandoc -s`
    \providecommand{\tightlist}{%
      \setlength{\itemsep}{0pt}\setlength{\parskip}{0pt}}
    \DefineVerbatimEnvironment{Highlighting}{Verbatim}{commandchars=\\\{\}}
    % Add ',fontsize=\small' for more characters per line
    \newenvironment{Shaded}{}{}
    \newcommand{\KeywordTok}[1]{\textcolor[rgb]{0.00,0.44,0.13}{\textbf{{#1}}}}
    \newcommand{\DataTypeTok}[1]{\textcolor[rgb]{0.56,0.13,0.00}{{#1}}}
    \newcommand{\DecValTok}[1]{\textcolor[rgb]{0.25,0.63,0.44}{{#1}}}
    \newcommand{\BaseNTok}[1]{\textcolor[rgb]{0.25,0.63,0.44}{{#1}}}
    \newcommand{\FloatTok}[1]{\textcolor[rgb]{0.25,0.63,0.44}{{#1}}}
    \newcommand{\CharTok}[1]{\textcolor[rgb]{0.25,0.44,0.63}{{#1}}}
    \newcommand{\StringTok}[1]{\textcolor[rgb]{0.25,0.44,0.63}{{#1}}}
    \newcommand{\CommentTok}[1]{\textcolor[rgb]{0.38,0.63,0.69}{\textit{{#1}}}}
    \newcommand{\OtherTok}[1]{\textcolor[rgb]{0.00,0.44,0.13}{{#1}}}
    \newcommand{\AlertTok}[1]{\textcolor[rgb]{1.00,0.00,0.00}{\textbf{{#1}}}}
    \newcommand{\FunctionTok}[1]{\textcolor[rgb]{0.02,0.16,0.49}{{#1}}}
    \newcommand{\RegionMarkerTok}[1]{{#1}}
    \newcommand{\ErrorTok}[1]{\textcolor[rgb]{1.00,0.00,0.00}{\textbf{{#1}}}}
    \newcommand{\NormalTok}[1]{{#1}}
    
    % Additional commands for more recent versions of Pandoc
    \newcommand{\ConstantTok}[1]{\textcolor[rgb]{0.53,0.00,0.00}{{#1}}}
    \newcommand{\SpecialCharTok}[1]{\textcolor[rgb]{0.25,0.44,0.63}{{#1}}}
    \newcommand{\VerbatimStringTok}[1]{\textcolor[rgb]{0.25,0.44,0.63}{{#1}}}
    \newcommand{\SpecialStringTok}[1]{\textcolor[rgb]{0.73,0.40,0.53}{{#1}}}
    \newcommand{\ImportTok}[1]{{#1}}
    \newcommand{\DocumentationTok}[1]{\textcolor[rgb]{0.73,0.13,0.13}{\textit{{#1}}}}
    \newcommand{\AnnotationTok}[1]{\textcolor[rgb]{0.38,0.63,0.69}{\textbf{\textit{{#1}}}}}
    \newcommand{\CommentVarTok}[1]{\textcolor[rgb]{0.38,0.63,0.69}{\textbf{\textit{{#1}}}}}
    \newcommand{\VariableTok}[1]{\textcolor[rgb]{0.10,0.09,0.49}{{#1}}}
    \newcommand{\ControlFlowTok}[1]{\textcolor[rgb]{0.00,0.44,0.13}{\textbf{{#1}}}}
    \newcommand{\OperatorTok}[1]{\textcolor[rgb]{0.40,0.40,0.40}{{#1}}}
    \newcommand{\BuiltInTok}[1]{{#1}}
    \newcommand{\ExtensionTok}[1]{{#1}}
    \newcommand{\PreprocessorTok}[1]{\textcolor[rgb]{0.74,0.48,0.00}{{#1}}}
    \newcommand{\AttributeTok}[1]{\textcolor[rgb]{0.49,0.56,0.16}{{#1}}}
    \newcommand{\InformationTok}[1]{\textcolor[rgb]{0.38,0.63,0.69}{\textbf{\textit{{#1}}}}}
    \newcommand{\WarningTok}[1]{\textcolor[rgb]{0.38,0.63,0.69}{\textbf{\textit{{#1}}}}}
    
    
    % Define a nice break command that doesn't care if a line doesn't already
    % exist.
    \def\br{\hspace*{\fill} \\* }
    % Math Jax compatability definitions
    \def\gt{>}
    \def\lt{<}
    % Document parameters
    \title{string\_manipulation}
    
    
    

    % Pygments definitions
    
\makeatletter
\def\PY@reset{\let\PY@it=\relax \let\PY@bf=\relax%
    \let\PY@ul=\relax \let\PY@tc=\relax%
    \let\PY@bc=\relax \let\PY@ff=\relax}
\def\PY@tok#1{\csname PY@tok@#1\endcsname}
\def\PY@toks#1+{\ifx\relax#1\empty\else%
    \PY@tok{#1}\expandafter\PY@toks\fi}
\def\PY@do#1{\PY@bc{\PY@tc{\PY@ul{%
    \PY@it{\PY@bf{\PY@ff{#1}}}}}}}
\def\PY#1#2{\PY@reset\PY@toks#1+\relax+\PY@do{#2}}

\expandafter\def\csname PY@tok@w\endcsname{\def\PY@tc##1{\textcolor[rgb]{0.73,0.73,0.73}{##1}}}
\expandafter\def\csname PY@tok@c\endcsname{\let\PY@it=\textit\def\PY@tc##1{\textcolor[rgb]{0.25,0.50,0.50}{##1}}}
\expandafter\def\csname PY@tok@cp\endcsname{\def\PY@tc##1{\textcolor[rgb]{0.74,0.48,0.00}{##1}}}
\expandafter\def\csname PY@tok@k\endcsname{\let\PY@bf=\textbf\def\PY@tc##1{\textcolor[rgb]{0.00,0.50,0.00}{##1}}}
\expandafter\def\csname PY@tok@kp\endcsname{\def\PY@tc##1{\textcolor[rgb]{0.00,0.50,0.00}{##1}}}
\expandafter\def\csname PY@tok@kt\endcsname{\def\PY@tc##1{\textcolor[rgb]{0.69,0.00,0.25}{##1}}}
\expandafter\def\csname PY@tok@o\endcsname{\def\PY@tc##1{\textcolor[rgb]{0.40,0.40,0.40}{##1}}}
\expandafter\def\csname PY@tok@ow\endcsname{\let\PY@bf=\textbf\def\PY@tc##1{\textcolor[rgb]{0.67,0.13,1.00}{##1}}}
\expandafter\def\csname PY@tok@nb\endcsname{\def\PY@tc##1{\textcolor[rgb]{0.00,0.50,0.00}{##1}}}
\expandafter\def\csname PY@tok@nf\endcsname{\def\PY@tc##1{\textcolor[rgb]{0.00,0.00,1.00}{##1}}}
\expandafter\def\csname PY@tok@nc\endcsname{\let\PY@bf=\textbf\def\PY@tc##1{\textcolor[rgb]{0.00,0.00,1.00}{##1}}}
\expandafter\def\csname PY@tok@nn\endcsname{\let\PY@bf=\textbf\def\PY@tc##1{\textcolor[rgb]{0.00,0.00,1.00}{##1}}}
\expandafter\def\csname PY@tok@ne\endcsname{\let\PY@bf=\textbf\def\PY@tc##1{\textcolor[rgb]{0.82,0.25,0.23}{##1}}}
\expandafter\def\csname PY@tok@nv\endcsname{\def\PY@tc##1{\textcolor[rgb]{0.10,0.09,0.49}{##1}}}
\expandafter\def\csname PY@tok@no\endcsname{\def\PY@tc##1{\textcolor[rgb]{0.53,0.00,0.00}{##1}}}
\expandafter\def\csname PY@tok@nl\endcsname{\def\PY@tc##1{\textcolor[rgb]{0.63,0.63,0.00}{##1}}}
\expandafter\def\csname PY@tok@ni\endcsname{\let\PY@bf=\textbf\def\PY@tc##1{\textcolor[rgb]{0.60,0.60,0.60}{##1}}}
\expandafter\def\csname PY@tok@na\endcsname{\def\PY@tc##1{\textcolor[rgb]{0.49,0.56,0.16}{##1}}}
\expandafter\def\csname PY@tok@nt\endcsname{\let\PY@bf=\textbf\def\PY@tc##1{\textcolor[rgb]{0.00,0.50,0.00}{##1}}}
\expandafter\def\csname PY@tok@nd\endcsname{\def\PY@tc##1{\textcolor[rgb]{0.67,0.13,1.00}{##1}}}
\expandafter\def\csname PY@tok@s\endcsname{\def\PY@tc##1{\textcolor[rgb]{0.73,0.13,0.13}{##1}}}
\expandafter\def\csname PY@tok@sd\endcsname{\let\PY@it=\textit\def\PY@tc##1{\textcolor[rgb]{0.73,0.13,0.13}{##1}}}
\expandafter\def\csname PY@tok@si\endcsname{\let\PY@bf=\textbf\def\PY@tc##1{\textcolor[rgb]{0.73,0.40,0.53}{##1}}}
\expandafter\def\csname PY@tok@se\endcsname{\let\PY@bf=\textbf\def\PY@tc##1{\textcolor[rgb]{0.73,0.40,0.13}{##1}}}
\expandafter\def\csname PY@tok@sr\endcsname{\def\PY@tc##1{\textcolor[rgb]{0.73,0.40,0.53}{##1}}}
\expandafter\def\csname PY@tok@ss\endcsname{\def\PY@tc##1{\textcolor[rgb]{0.10,0.09,0.49}{##1}}}
\expandafter\def\csname PY@tok@sx\endcsname{\def\PY@tc##1{\textcolor[rgb]{0.00,0.50,0.00}{##1}}}
\expandafter\def\csname PY@tok@m\endcsname{\def\PY@tc##1{\textcolor[rgb]{0.40,0.40,0.40}{##1}}}
\expandafter\def\csname PY@tok@gh\endcsname{\let\PY@bf=\textbf\def\PY@tc##1{\textcolor[rgb]{0.00,0.00,0.50}{##1}}}
\expandafter\def\csname PY@tok@gu\endcsname{\let\PY@bf=\textbf\def\PY@tc##1{\textcolor[rgb]{0.50,0.00,0.50}{##1}}}
\expandafter\def\csname PY@tok@gd\endcsname{\def\PY@tc##1{\textcolor[rgb]{0.63,0.00,0.00}{##1}}}
\expandafter\def\csname PY@tok@gi\endcsname{\def\PY@tc##1{\textcolor[rgb]{0.00,0.63,0.00}{##1}}}
\expandafter\def\csname PY@tok@gr\endcsname{\def\PY@tc##1{\textcolor[rgb]{1.00,0.00,0.00}{##1}}}
\expandafter\def\csname PY@tok@ge\endcsname{\let\PY@it=\textit}
\expandafter\def\csname PY@tok@gs\endcsname{\let\PY@bf=\textbf}
\expandafter\def\csname PY@tok@gp\endcsname{\let\PY@bf=\textbf\def\PY@tc##1{\textcolor[rgb]{0.00,0.00,0.50}{##1}}}
\expandafter\def\csname PY@tok@go\endcsname{\def\PY@tc##1{\textcolor[rgb]{0.53,0.53,0.53}{##1}}}
\expandafter\def\csname PY@tok@gt\endcsname{\def\PY@tc##1{\textcolor[rgb]{0.00,0.27,0.87}{##1}}}
\expandafter\def\csname PY@tok@err\endcsname{\def\PY@bc##1{\setlength{\fboxsep}{0pt}\fcolorbox[rgb]{1.00,0.00,0.00}{1,1,1}{\strut ##1}}}
\expandafter\def\csname PY@tok@kc\endcsname{\let\PY@bf=\textbf\def\PY@tc##1{\textcolor[rgb]{0.00,0.50,0.00}{##1}}}
\expandafter\def\csname PY@tok@kd\endcsname{\let\PY@bf=\textbf\def\PY@tc##1{\textcolor[rgb]{0.00,0.50,0.00}{##1}}}
\expandafter\def\csname PY@tok@kn\endcsname{\let\PY@bf=\textbf\def\PY@tc##1{\textcolor[rgb]{0.00,0.50,0.00}{##1}}}
\expandafter\def\csname PY@tok@kr\endcsname{\let\PY@bf=\textbf\def\PY@tc##1{\textcolor[rgb]{0.00,0.50,0.00}{##1}}}
\expandafter\def\csname PY@tok@bp\endcsname{\def\PY@tc##1{\textcolor[rgb]{0.00,0.50,0.00}{##1}}}
\expandafter\def\csname PY@tok@fm\endcsname{\def\PY@tc##1{\textcolor[rgb]{0.00,0.00,1.00}{##1}}}
\expandafter\def\csname PY@tok@vc\endcsname{\def\PY@tc##1{\textcolor[rgb]{0.10,0.09,0.49}{##1}}}
\expandafter\def\csname PY@tok@vg\endcsname{\def\PY@tc##1{\textcolor[rgb]{0.10,0.09,0.49}{##1}}}
\expandafter\def\csname PY@tok@vi\endcsname{\def\PY@tc##1{\textcolor[rgb]{0.10,0.09,0.49}{##1}}}
\expandafter\def\csname PY@tok@vm\endcsname{\def\PY@tc##1{\textcolor[rgb]{0.10,0.09,0.49}{##1}}}
\expandafter\def\csname PY@tok@sa\endcsname{\def\PY@tc##1{\textcolor[rgb]{0.73,0.13,0.13}{##1}}}
\expandafter\def\csname PY@tok@sb\endcsname{\def\PY@tc##1{\textcolor[rgb]{0.73,0.13,0.13}{##1}}}
\expandafter\def\csname PY@tok@sc\endcsname{\def\PY@tc##1{\textcolor[rgb]{0.73,0.13,0.13}{##1}}}
\expandafter\def\csname PY@tok@dl\endcsname{\def\PY@tc##1{\textcolor[rgb]{0.73,0.13,0.13}{##1}}}
\expandafter\def\csname PY@tok@s2\endcsname{\def\PY@tc##1{\textcolor[rgb]{0.73,0.13,0.13}{##1}}}
\expandafter\def\csname PY@tok@sh\endcsname{\def\PY@tc##1{\textcolor[rgb]{0.73,0.13,0.13}{##1}}}
\expandafter\def\csname PY@tok@s1\endcsname{\def\PY@tc##1{\textcolor[rgb]{0.73,0.13,0.13}{##1}}}
\expandafter\def\csname PY@tok@mb\endcsname{\def\PY@tc##1{\textcolor[rgb]{0.40,0.40,0.40}{##1}}}
\expandafter\def\csname PY@tok@mf\endcsname{\def\PY@tc##1{\textcolor[rgb]{0.40,0.40,0.40}{##1}}}
\expandafter\def\csname PY@tok@mh\endcsname{\def\PY@tc##1{\textcolor[rgb]{0.40,0.40,0.40}{##1}}}
\expandafter\def\csname PY@tok@mi\endcsname{\def\PY@tc##1{\textcolor[rgb]{0.40,0.40,0.40}{##1}}}
\expandafter\def\csname PY@tok@il\endcsname{\def\PY@tc##1{\textcolor[rgb]{0.40,0.40,0.40}{##1}}}
\expandafter\def\csname PY@tok@mo\endcsname{\def\PY@tc##1{\textcolor[rgb]{0.40,0.40,0.40}{##1}}}
\expandafter\def\csname PY@tok@ch\endcsname{\let\PY@it=\textit\def\PY@tc##1{\textcolor[rgb]{0.25,0.50,0.50}{##1}}}
\expandafter\def\csname PY@tok@cm\endcsname{\let\PY@it=\textit\def\PY@tc##1{\textcolor[rgb]{0.25,0.50,0.50}{##1}}}
\expandafter\def\csname PY@tok@cpf\endcsname{\let\PY@it=\textit\def\PY@tc##1{\textcolor[rgb]{0.25,0.50,0.50}{##1}}}
\expandafter\def\csname PY@tok@c1\endcsname{\let\PY@it=\textit\def\PY@tc##1{\textcolor[rgb]{0.25,0.50,0.50}{##1}}}
\expandafter\def\csname PY@tok@cs\endcsname{\let\PY@it=\textit\def\PY@tc##1{\textcolor[rgb]{0.25,0.50,0.50}{##1}}}

\def\PYZbs{\char`\\}
\def\PYZus{\char`\_}
\def\PYZob{\char`\{}
\def\PYZcb{\char`\}}
\def\PYZca{\char`\^}
\def\PYZam{\char`\&}
\def\PYZlt{\char`\<}
\def\PYZgt{\char`\>}
\def\PYZsh{\char`\#}
\def\PYZpc{\char`\%}
\def\PYZdl{\char`\$}
\def\PYZhy{\char`\-}
\def\PYZsq{\char`\'}
\def\PYZdq{\char`\"}
\def\PYZti{\char`\~}
% for compatibility with earlier versions
\def\PYZat{@}
\def\PYZlb{[}
\def\PYZrb{]}
\makeatother


    % Exact colors from NB
    \definecolor{incolor}{rgb}{0.0, 0.0, 0.5}
    \definecolor{outcolor}{rgb}{0.545, 0.0, 0.0}



    
    % Prevent overflowing lines due to hard-to-break entities
    \sloppy 
    % Setup hyperref package
    \hypersetup{
      breaklinks=true,  % so long urls are correctly broken across lines
      colorlinks=true,
      urlcolor=urlcolor,
      linkcolor=linkcolor,
      citecolor=citecolor,
      }
    % Slightly bigger margins than the latex defaults
    
    \geometry{verbose,tmargin=1in,bmargin=1in,lmargin=1in,rmargin=1in}
    
    

    \begin{document}
    
    
    \maketitle
    
    

    
    \hypertarget{string-manipulation}{%
\section{String Manipulation}\label{string-manipulation}}

\begin{itemize}
\tightlist
\item
  Text is the most common form of data your program will handle
\item
  Previously: we've concatenated two strings together using the
  \texttt{+}
\item
  Now: Extract partial strings from string values

  \begin{itemize}
  \tightlist
  \item
    Add or remove spacing
  \item
    convert letters to lowercase or uppercase
  \item
    check formatting of strings
  \end{itemize}
\item
  Write code to accesss the clipboard and paste text
\end{itemize}

    \hypertarget{strings}{%
\subsection{Strings}\label{strings}}

    \begin{Verbatim}[commandchars=\\\{\}]
{\color{incolor}In [{\color{incolor}2}]:} \PY{n+nb}{print}\PY{p}{(}\PY{l+s+s1}{\PYZsq{}}\PY{l+s+s1}{This is a string}\PY{l+s+s1}{\PYZsq{}}\PY{p}{)} \PY{c+c1}{\PYZsh{} Single quotes}
\end{Verbatim}


    \begin{Verbatim}[commandchars=\\\{\}]
This is a string

    \end{Verbatim}

    \begin{Verbatim}[commandchars=\\\{\}]
{\color{incolor}In [{\color{incolor}3}]:} \PY{n+nb}{print}\PY{p}{(}\PY{l+s+s2}{\PYZdq{}}\PY{l+s+s2}{This is }\PY{l+s+s2}{\PYZsq{}}\PY{l+s+s2}{also a string}\PY{l+s+s2}{\PYZdq{}}\PY{p}{)} \PY{c+c1}{\PYZsh{} Double quotes}
\end{Verbatim}


    \begin{Verbatim}[commandchars=\\\{\}]
This is also a string

    \end{Verbatim}

    \hypertarget{escape-characters}{%
\subsection{Escape Characters}\label{escape-characters}}

\begin{itemize}
\tightlist
\item
  \textbf{Escape characters}: allow you to use characters that are
  otherwise impossible to put into a string
\item
  An escape character consists of a backslah (\textbackslash{}) followed
  by the character you want to add to the string
\end{itemize}

    \begin{Verbatim}[commandchars=\\\{\}]
{\color{incolor}In [{\color{incolor}51}]:} \PY{n+nb}{print}\PY{p}{(}\PY{l+s+s1}{\PYZsq{}}\PY{l+s+s1}{Say hello to Eli}\PY{l+s+se}{\PYZbs{}\PYZsq{}}\PY{l+s+s1}{s dog}\PY{l+s+s1}{\PYZsq{}}\PY{p}{)}
\end{Verbatim}


    \begin{Verbatim}[commandchars=\\\{\}]
Say hello to Eli's dog

    \end{Verbatim}

    \begin{itemize}
\tightlist
\item
  Since the single quote in Eli\textbackslash{}'s has a backslash, it is
  not a single quote ment to end the string value.
\item
  The escape character \textbackslash{}' and \textbackslash{}" let you
  put quotes inside your strings.
\end{itemize}

    \begin{longtable}[]{@{}ll@{}}
\toprule
\textbf{Escape character} & \textbf{Prints as}\tabularnewline
\midrule
\endhead
\textbackslash{}' & single quote\tabularnewline
\textbackslash{}" & double quote\tabularnewline
\t & tab\tabularnewline
\n & newline\tabularnewline
\textbackslash{}\textbackslash{} & backslash\tabularnewline
\bottomrule
\end{longtable}

    \begin{Verbatim}[commandchars=\\\{\}]
{\color{incolor}In [{\color{incolor}5}]:} \PY{n+nb}{print}\PY{p}{(}\PY{l+s+s2}{\PYZdq{}}\PY{l+s+s2}{Dear Bob,}\PY{l+s+se}{\PYZbs{}n}\PY{l+s+s2}{This is a newline.}\PY{l+s+se}{\PYZbs{}n}\PY{l+s+s2}{Here is another line.}\PY{l+s+s2}{\PYZdq{}}\PY{p}{)}
\end{Verbatim}


    \begin{Verbatim}[commandchars=\\\{\}]
Dear Bob,
This is a newline.
Here is another line.

    \end{Verbatim}

    \hypertarget{raw-strings}{%
\subsection{Raw Strings}\label{raw-strings}}

\begin{itemize}
\tightlist
\item
  Place an \texttt{r} before the beginning quote mark of a string to
  make it a raw string
\item
  \textbf{Raw string} completely ignores all excape characters and
  prints any backslash that appears in the string.
\end{itemize}

    \begin{Verbatim}[commandchars=\\\{\}]
{\color{incolor}In [{\color{incolor}6}]:} \PY{n+nb}{print}\PY{p}{(}\PY{l+s+sa}{r}\PY{l+s+s1}{\PYZsq{}}\PY{l+s+s1}{Say hello to Eli}\PY{l+s+se}{\PYZbs{}\PYZsq{}}\PY{l+s+s1}{s dog}\PY{l+s+s1}{\PYZsq{}}\PY{p}{)}
\end{Verbatim}


    \begin{Verbatim}[commandchars=\\\{\}]
Say hello to Eli\textbackslash{}'s dog

    \end{Verbatim}

    \begin{itemize}
\tightlist
\item
  Raw strings are helpful if you are typing string values that contain
  many backslashes, such as the strings used for regular expressions
\end{itemize}

    \hypertarget{multiline-strings}{%
\subsection{Multiline Strings}\label{multiline-strings}}

    \begin{Verbatim}[commandchars=\\\{\}]
{\color{incolor}In [{\color{incolor}4}]:} \PY{n+nb}{print}\PY{p}{(}\PY{l+s+s2}{\PYZdq{}\PYZdq{}\PYZdq{}}\PY{l+s+s2}{Dear }\PY{l+s+si}{\PYZob{}name\PYZcb{}}\PY{l+s+s2}{,}
        \PY{l+s+s2}{I would like to apply to }\PY{l+s+si}{\PYZob{}loc\PYZcb{}}\PY{l+s+s2}{.}
        \PY{l+s+s2}{Here is another line.}
        
        \PY{l+s+s2}{Thank you.}
        
        \PY{l+s+s2}{Markum}\PY{l+s+s2}{\PYZdq{}\PYZdq{}\PYZdq{}}\PY{o}{.}\PY{n}{format}\PY{p}{(}\PY{n}{name}\PY{o}{=}\PY{l+s+s1}{\PYZsq{}}\PY{l+s+s1}{Mr. Bill}\PY{l+s+s1}{\PYZsq{}}\PY{p}{,} \PY{n}{loc}\PY{o}{=}\PY{l+s+s1}{\PYZsq{}}\PY{l+s+s1}{HSU \PYZhy{} School of Business}\PY{l+s+s1}{\PYZsq{}}\PY{p}{)}\PY{p}{)}
\end{Verbatim}


    \begin{Verbatim}[commandchars=\\\{\}]
Dear Mr. Bill,
I would like to apply to HSU - School of Business.
Here is another line.

Thank you.

Markum

    \end{Verbatim}

    \hypertarget{in-and-not-in-operators-with-strings}{%
\subsection{in and not in Operators with
Strings}\label{in-and-not-in-operators-with-strings}}

    \begin{Verbatim}[commandchars=\\\{\}]
{\color{incolor}In [{\color{incolor}10}]:} \PY{l+s+s1}{\PYZsq{}}\PY{l+s+s1}{Hello}\PY{l+s+s1}{\PYZsq{}} \PY{o+ow}{in} \PY{l+s+s1}{\PYZsq{}}\PY{l+s+s1}{Hello world}\PY{l+s+s1}{\PYZsq{}}
\end{Verbatim}


\begin{Verbatim}[commandchars=\\\{\}]
{\color{outcolor}Out[{\color{outcolor}10}]:} True
\end{Verbatim}
            
    \begin{Verbatim}[commandchars=\\\{\}]
{\color{incolor}In [{\color{incolor}12}]:} \PY{l+s+s1}{\PYZsq{}}\PY{l+s+s1}{HELLO}\PY{l+s+s1}{\PYZsq{}} \PY{o+ow}{in} \PY{l+s+s1}{\PYZsq{}}\PY{l+s+s1}{Hello world}\PY{l+s+s1}{\PYZsq{}}
\end{Verbatim}


\begin{Verbatim}[commandchars=\\\{\}]
{\color{outcolor}Out[{\color{outcolor}12}]:} False
\end{Verbatim}
            
    \begin{Verbatim}[commandchars=\\\{\}]
{\color{incolor}In [{\color{incolor}13}]:} \PY{l+s+s1}{\PYZsq{}}\PY{l+s+s1}{Hello}\PY{l+s+s1}{\PYZsq{}} \PY{o+ow}{not} \PY{o+ow}{in} \PY{l+s+s1}{\PYZsq{}}\PY{l+s+s1}{Hello world}\PY{l+s+s1}{\PYZsq{}}
\end{Verbatim}


\begin{Verbatim}[commandchars=\\\{\}]
{\color{outcolor}Out[{\color{outcolor}13}]:} False
\end{Verbatim}
            
    \hypertarget{string-methods}{%
\subsection{String Methods}\label{string-methods}}

    \begin{Verbatim}[commandchars=\\\{\}]
{\color{incolor}In [{\color{incolor}14}]:} \PY{n}{s} \PY{o}{=} \PY{l+s+s1}{\PYZsq{}}\PY{l+s+s1}{Hello World}\PY{l+s+s1}{\PYZsq{}}
\end{Verbatim}


    \begin{Verbatim}[commandchars=\\\{\}]
{\color{incolor}In [{\color{incolor}15}]:} \PY{n}{s}\PY{o}{.}\PY{n}{upper}\PY{p}{(}\PY{p}{)}
\end{Verbatim}


\begin{Verbatim}[commandchars=\\\{\}]
{\color{outcolor}Out[{\color{outcolor}15}]:} 'HELLO WORLD'
\end{Verbatim}
            
    \begin{Verbatim}[commandchars=\\\{\}]
{\color{incolor}In [{\color{incolor}18}]:} \PY{n}{s}\PY{o}{.}\PY{n}{lower}\PY{p}{(}\PY{p}{)}
\end{Verbatim}


\begin{Verbatim}[commandchars=\\\{\}]
{\color{outcolor}Out[{\color{outcolor}18}]:} 'hello world'
\end{Verbatim}
            
    \begin{Verbatim}[commandchars=\\\{\}]
{\color{incolor}In [{\color{incolor}19}]:} \PY{n}{s\PYZus{}2} \PY{o}{=} \PY{l+s+s1}{\PYZsq{}}\PY{l+s+s1}{here is another sentence.}\PY{l+s+s1}{\PYZsq{}}
\end{Verbatim}


    \begin{Verbatim}[commandchars=\\\{\}]
{\color{incolor}In [{\color{incolor}20}]:} \PY{n}{s\PYZus{}2}\PY{o}{.}\PY{n}{title}\PY{p}{(}\PY{p}{)}
\end{Verbatim}


\begin{Verbatim}[commandchars=\\\{\}]
{\color{outcolor}Out[{\color{outcolor}20}]:} 'Here Is Another Sentence.'
\end{Verbatim}
            
    \begin{Verbatim}[commandchars=\\\{\}]
{\color{incolor}In [{\color{incolor}8}]:} \PY{n}{feeling} \PY{o}{=} \PY{n+nb}{input}\PY{p}{(}\PY{l+s+s1}{\PYZsq{}}\PY{l+s+s1}{How are you feeling?}\PY{l+s+se}{\PYZbs{}n}\PY{l+s+s1}{\PYZsq{}}\PY{p}{)}
        \PY{k}{if} \PY{n}{feeling}\PY{o}{.}\PY{n}{lower}\PY{p}{(}\PY{p}{)} \PY{o}{==} \PY{l+s+s1}{\PYZsq{}}\PY{l+s+s1}{great}\PY{l+s+s1}{\PYZsq{}}\PY{p}{:}
            \PY{n+nb}{print}\PY{p}{(}\PY{l+s+s1}{\PYZsq{}}\PY{l+s+s1}{I feel great too.}\PY{l+s+s1}{\PYZsq{}}\PY{p}{)}
        \PY{k}{else}\PY{p}{:}
            \PY{n+nb}{print}\PY{p}{(}\PY{l+s+s1}{\PYZsq{}}\PY{l+s+s1}{I hope the rest of your day is good.}\PY{l+s+s1}{\PYZsq{}}\PY{p}{)}
\end{Verbatim}


    \begin{Verbatim}[commandchars=\\\{\}]
How are you feeling?
 Great
I hope the rest of your day is good.

    \end{Verbatim}

    \hypertarget{isx-methods}{%
\subsection{isX Methods}\label{isx-methods}}

    \begin{Verbatim}[commandchars=\\\{\}]
{\color{incolor}In [{\color{incolor}30}]:} \PY{l+s+s1}{\PYZsq{}}\PY{l+s+s1}{hello}\PY{l+s+s1}{\PYZsq{}}\PY{o}{.}\PY{n}{islower}\PY{p}{(}\PY{p}{)}
\end{Verbatim}


\begin{Verbatim}[commandchars=\\\{\}]
{\color{outcolor}Out[{\color{outcolor}30}]:} True
\end{Verbatim}
            
    \begin{Verbatim}[commandchars=\\\{\}]
{\color{incolor}In [{\color{incolor}31}]:} \PY{l+s+s1}{\PYZsq{}}\PY{l+s+s1}{hello}\PY{l+s+s1}{\PYZsq{}}\PY{o}{.}\PY{n}{isupper}\PY{p}{(}\PY{p}{)}
\end{Verbatim}


\begin{Verbatim}[commandchars=\\\{\}]
{\color{outcolor}Out[{\color{outcolor}31}]:} False
\end{Verbatim}
            
    \begin{Verbatim}[commandchars=\\\{\}]
{\color{incolor}In [{\color{incolor}24}]:} \PY{l+s+s1}{\PYZsq{}}\PY{l+s+s1}{hello}\PY{l+s+s1}{\PYZsq{}}\PY{o}{.}\PY{n}{isalpha}\PY{p}{(}\PY{p}{)}
\end{Verbatim}


\begin{Verbatim}[commandchars=\\\{\}]
{\color{outcolor}Out[{\color{outcolor}24}]:} True
\end{Verbatim}
            
    \begin{Verbatim}[commandchars=\\\{\}]
{\color{incolor}In [{\color{incolor}34}]:} \PY{l+s+s1}{\PYZsq{}}\PY{l+s+s1}{hello123}\PY{l+s+s1}{\PYZsq{}}\PY{o}{.}\PY{n}{isalnum}\PY{p}{(}\PY{p}{)}
\end{Verbatim}


\begin{Verbatim}[commandchars=\\\{\}]
{\color{outcolor}Out[{\color{outcolor}34}]:} True
\end{Verbatim}
            
    \begin{Verbatim}[commandchars=\\\{\}]
{\color{incolor}In [{\color{incolor}35}]:} \PY{l+s+s1}{\PYZsq{}}\PY{l+s+s1}{hello}\PY{l+s+s1}{\PYZsq{}}\PY{o}{.}\PY{n}{isalnum}\PY{p}{(}\PY{p}{)}
\end{Verbatim}


\begin{Verbatim}[commandchars=\\\{\}]
{\color{outcolor}Out[{\color{outcolor}35}]:} True
\end{Verbatim}
            
    \begin{Verbatim}[commandchars=\\\{\}]
{\color{incolor}In [{\color{incolor}36}]:} \PY{l+s+s1}{\PYZsq{}}\PY{l+s+s1}{123}\PY{l+s+s1}{\PYZsq{}}\PY{o}{.}\PY{n}{isdecimal}\PY{p}{(}\PY{p}{)}
\end{Verbatim}


\begin{Verbatim}[commandchars=\\\{\}]
{\color{outcolor}Out[{\color{outcolor}36}]:} True
\end{Verbatim}
            
    \begin{Verbatim}[commandchars=\\\{\}]
{\color{incolor}In [{\color{incolor}37}]:} \PY{l+s+s1}{\PYZsq{}}\PY{l+s+s1}{ }\PY{l+s+s1}{\PYZsq{}}\PY{o}{.}\PY{n}{isspace}\PY{p}{(}\PY{p}{)}
\end{Verbatim}


\begin{Verbatim}[commandchars=\\\{\}]
{\color{outcolor}Out[{\color{outcolor}37}]:} True
\end{Verbatim}
            
    \begin{Verbatim}[commandchars=\\\{\}]
{\color{incolor}In [{\color{incolor}38}]:} \PY{l+s+s1}{\PYZsq{}}\PY{l+s+s1}{This Is A Title}\PY{l+s+s1}{\PYZsq{}}\PY{o}{.}\PY{n}{istitle}\PY{p}{(}\PY{p}{)}
\end{Verbatim}


\begin{Verbatim}[commandchars=\\\{\}]
{\color{outcolor}Out[{\color{outcolor}38}]:} True
\end{Verbatim}
            
    \begin{Verbatim}[commandchars=\\\{\}]
{\color{incolor}In [{\color{incolor}39}]:} \PY{k}{while} \PY{k+kc}{True}\PY{p}{:}
             \PY{n+nb}{print}\PY{p}{(}\PY{l+s+s1}{\PYZsq{}}\PY{l+s+s1}{Enter your age:}\PY{l+s+s1}{\PYZsq{}}\PY{p}{)}
             \PY{n}{age} \PY{o}{=} \PY{n+nb}{input}\PY{p}{(}\PY{p}{)}
             \PY{k}{if} \PY{n}{age}\PY{o}{.}\PY{n}{isdecimal}\PY{p}{(}\PY{p}{)}\PY{p}{:}
                 \PY{k}{break}
             \PY{n+nb}{print}\PY{p}{(}\PY{l+s+s1}{\PYZsq{}}\PY{l+s+s1}{Please enter a number for your age.}\PY{l+s+s1}{\PYZsq{}}\PY{p}{)}
\end{Verbatim}


    \begin{Verbatim}[commandchars=\\\{\}]
Enter your age:
five
Please enter a number for your age.
Enter your age:
6

    \end{Verbatim}

    \hypertarget{join-and-split-methods}{%
\subsection{join and split methods}\label{join-and-split-methods}}

    \begin{itemize}
\tightlist
\item
  \texttt{join()} method is useful when you have a list of strings that
  need to be joined together into a single string value
\end{itemize}

    \begin{Verbatim}[commandchars=\\\{\}]
{\color{incolor}In [{\color{incolor}13}]:} \PY{n}{a\PYZus{}list} \PY{o}{=} \PY{p}{[}\PY{l+s+s1}{\PYZsq{}}\PY{l+s+s1}{this}\PY{l+s+s1}{\PYZsq{}}\PY{p}{,}\PY{l+s+s1}{\PYZsq{}}\PY{l+s+s1}{is}\PY{l+s+s1}{\PYZsq{}}\PY{p}{,}\PY{l+s+s1}{\PYZsq{}}\PY{l+s+s1}{a}\PY{l+s+s1}{\PYZsq{}}\PY{p}{,}\PY{l+s+s1}{\PYZsq{}}\PY{l+s+s1}{list}\PY{l+s+s1}{\PYZsq{}}\PY{p}{]}
\end{Verbatim}


    \begin{Verbatim}[commandchars=\\\{\}]
{\color{incolor}In [{\color{incolor}44}]:} \PY{l+s+s1}{\PYZsq{}}\PY{l+s+s1}{, }\PY{l+s+s1}{\PYZsq{}}\PY{o}{.}\PY{n}{join}\PY{p}{(}\PY{n}{a\PYZus{}list}\PY{p}{)}
\end{Verbatim}


\begin{Verbatim}[commandchars=\\\{\}]
{\color{outcolor}Out[{\color{outcolor}44}]:} 'this, is, a, list'
\end{Verbatim}
            
    \begin{Verbatim}[commandchars=\\\{\}]
{\color{incolor}In [{\color{incolor}45}]:} \PY{l+s+s1}{\PYZsq{}}\PY{l+s+s1}{ }\PY{l+s+s1}{\PYZsq{}}\PY{o}{.}\PY{n}{join}\PY{p}{(}\PY{n}{a\PYZus{}list}\PY{p}{)}
\end{Verbatim}


\begin{Verbatim}[commandchars=\\\{\}]
{\color{outcolor}Out[{\color{outcolor}45}]:} 'this is a list'
\end{Verbatim}
            
    \begin{Verbatim}[commandchars=\\\{\}]
{\color{incolor}In [{\color{incolor}16}]:} \PY{l+s+s1}{\PYZsq{}}\PY{l+s+s1}{\PYZus{}}\PY{l+s+s1}{\PYZsq{}}\PY{o}{.}\PY{n}{join}\PY{p}{(}\PY{n}{a\PYZus{}list}\PY{p}{)}
\end{Verbatim}


\begin{Verbatim}[commandchars=\\\{\}]
{\color{outcolor}Out[{\color{outcolor}16}]:} 'this\_is\_a\_list'
\end{Verbatim}
            
    \begin{itemize}
\tightlist
\item
  \texttt{split()} method does the opposite of join
\end{itemize}

    \begin{Verbatim}[commandchars=\\\{\}]
{\color{incolor}In [{\color{incolor}49}]:} \PY{l+s+s1}{\PYZsq{}}\PY{l+s+s1}{Here is a string}\PY{l+s+s1}{\PYZsq{}}\PY{o}{.}\PY{n}{split}\PY{p}{(}\PY{p}{)} \PY{c+c1}{\PYZsh{} Default is spaces}
\end{Verbatim}


\begin{Verbatim}[commandchars=\\\{\}]
{\color{outcolor}Out[{\color{outcolor}49}]:} ['Here', 'is', 'a', 'string']
\end{Verbatim}
            
    \begin{Verbatim}[commandchars=\\\{\}]
{\color{incolor}In [{\color{incolor}50}]:} \PY{l+s+s1}{\PYZsq{}}\PY{l+s+s1}{thisABCDEisABCDEaABCDElist}\PY{l+s+s1}{\PYZsq{}}\PY{o}{.}\PY{n}{split}\PY{p}{(}\PY{l+s+s1}{\PYZsq{}}\PY{l+s+s1}{ABCDE}\PY{l+s+s1}{\PYZsq{}}\PY{p}{)}
\end{Verbatim}


\begin{Verbatim}[commandchars=\\\{\}]
{\color{outcolor}Out[{\color{outcolor}50}]:} ['this', 'is', 'a', 'list']
\end{Verbatim}
            
    \begin{Verbatim}[commandchars=\\\{\}]
{\color{incolor}In [{\color{incolor}55}]:} \PY{n}{s} \PY{o}{=} \PY{l+s+s2}{\PYZdq{}\PYZdq{}\PYZdq{}}\PY{l+s+s2}{Dear Bob,}
         \PY{l+s+s2}{This is a newline.}
         \PY{l+s+s2}{Here is another line.}
         \PY{l+s+s2}{\PYZdq{}\PYZdq{}\PYZdq{}}
\end{Verbatim}


    \begin{Verbatim}[commandchars=\\\{\}]
{\color{incolor}In [{\color{incolor}56}]:} \PY{n}{s}\PY{o}{.}\PY{n}{split}\PY{p}{(}\PY{l+s+s1}{\PYZsq{}}\PY{l+s+se}{\PYZbs{}n}\PY{l+s+s1}{\PYZsq{}}\PY{p}{)}
\end{Verbatim}


\begin{Verbatim}[commandchars=\\\{\}]
{\color{outcolor}Out[{\color{outcolor}56}]:} ['Dear Bob,', 'This is a newline.', 'Here is another line.', '']
\end{Verbatim}
            
    \hypertarget{justifying-text}{%
\subsection{Justifying Text}\label{justifying-text}}

\begin{itemize}
\tightlist
\item
  \texttt{rjust()}
\item
  \texttt{ljust()}
\item
  \texttt{center()}
\item
  The first argument to both methods is an integer length for the
  justified string.
\end{itemize}

    \begin{Verbatim}[commandchars=\\\{\}]
{\color{incolor}In [{\color{incolor}57}]:} \PY{l+s+s1}{\PYZsq{}}\PY{l+s+s1}{hello}\PY{l+s+s1}{\PYZsq{}}\PY{o}{.}\PY{n}{rjust}\PY{p}{(}\PY{l+m+mi}{10}\PY{p}{)}
\end{Verbatim}


\begin{Verbatim}[commandchars=\\\{\}]
{\color{outcolor}Out[{\color{outcolor}57}]:} '     hello'
\end{Verbatim}
            
    \begin{Verbatim}[commandchars=\\\{\}]
{\color{incolor}In [{\color{incolor}58}]:} \PY{l+s+s1}{\PYZsq{}}\PY{l+s+s1}{hello}\PY{l+s+s1}{\PYZsq{}}\PY{o}{.}\PY{n}{rjust}\PY{p}{(}\PY{l+m+mi}{20}\PY{p}{)}
\end{Verbatim}


\begin{Verbatim}[commandchars=\\\{\}]
{\color{outcolor}Out[{\color{outcolor}58}]:} '               hello'
\end{Verbatim}
            
    \begin{Verbatim}[commandchars=\\\{\}]
{\color{incolor}In [{\color{incolor}59}]:} \PY{l+s+s1}{\PYZsq{}}\PY{l+s+s1}{hello}\PY{l+s+s1}{\PYZsq{}}\PY{o}{.}\PY{n}{ljust}\PY{p}{(}\PY{l+m+mi}{10}\PY{p}{)}
\end{Verbatim}


\begin{Verbatim}[commandchars=\\\{\}]
{\color{outcolor}Out[{\color{outcolor}59}]:} 'hello     '
\end{Verbatim}
            
    \begin{Verbatim}[commandchars=\\\{\}]
{\color{incolor}In [{\color{incolor}60}]:} \PY{l+s+s1}{\PYZsq{}}\PY{l+s+s1}{hello}\PY{l+s+s1}{\PYZsq{}}\PY{o}{.}\PY{n}{center}\PY{p}{(}\PY{l+m+mi}{20}\PY{p}{)}
\end{Verbatim}


\begin{Verbatim}[commandchars=\\\{\}]
{\color{outcolor}Out[{\color{outcolor}60}]:} '       hello        '
\end{Verbatim}
            
    \begin{Verbatim}[commandchars=\\\{\}]
{\color{incolor}In [{\color{incolor}61}]:} \PY{l+s+s1}{\PYZsq{}}\PY{l+s+s1}{hello}\PY{l+s+s1}{\PYZsq{}}\PY{o}{.}\PY{n}{rjust}\PY{p}{(}\PY{l+m+mi}{10}\PY{p}{,} \PY{l+s+s2}{\PYZdq{}}\PY{l+s+s2}{\PYZhy{}}\PY{l+s+s2}{\PYZdq{}}\PY{p}{)}
\end{Verbatim}


\begin{Verbatim}[commandchars=\\\{\}]
{\color{outcolor}Out[{\color{outcolor}61}]:} '-----hello'
\end{Verbatim}
            
    \begin{Verbatim}[commandchars=\\\{\}]
{\color{incolor}In [{\color{incolor}63}]:} \PY{l+s+s1}{\PYZsq{}}\PY{l+s+s1}{hello}\PY{l+s+s1}{\PYZsq{}}\PY{o}{.}\PY{n}{center}\PY{p}{(}\PY{l+m+mi}{20}\PY{p}{,}\PY{l+s+s2}{\PYZdq{}}\PY{l+s+s2}{\PYZam{}}\PY{l+s+s2}{\PYZdq{}}\PY{p}{)}
\end{Verbatim}


\begin{Verbatim}[commandchars=\\\{\}]
{\color{outcolor}Out[{\color{outcolor}63}]:} '\&\&\&\&\&\&\&hello\&\&\&\&\&\&\&\&'
\end{Verbatim}
            
    \begin{Verbatim}[commandchars=\\\{\}]
{\color{incolor}In [{\color{incolor}17}]:} \PY{k}{def} \PY{n+nf}{price\PYZus{}board}\PY{p}{(}\PY{n}{items\PYZus{}dict}\PY{p}{,} \PY{n}{left\PYZus{}width}\PY{p}{,} \PY{n}{right\PYZus{}width}\PY{p}{)}\PY{p}{:}
             \PY{n+nb}{print}\PY{p}{(}\PY{l+s+s1}{\PYZsq{}}\PY{l+s+s1}{Breakfast Specials}\PY{l+s+s1}{\PYZsq{}}\PY{o}{.}\PY{n}{center}\PY{p}{(}\PY{n}{left\PYZus{}width} \PY{o}{+} \PY{n}{right\PYZus{}width}\PY{p}{,} \PY{l+s+s1}{\PYZsq{}}\PY{l+s+s1}{=}\PY{l+s+s1}{\PYZsq{}}\PY{p}{)}\PY{p}{)}
             \PY{k}{for} \PY{n}{k}\PY{p}{,} \PY{n}{v} \PY{o+ow}{in} \PY{n}{items\PYZus{}dict}\PY{o}{.}\PY{n}{items}\PY{p}{(}\PY{p}{)}\PY{p}{:}
                 \PY{n+nb}{print}\PY{p}{(}\PY{n}{k}\PY{o}{.}\PY{n}{ljust}\PY{p}{(}\PY{n}{left\PYZus{}width}\PY{p}{,} \PY{l+s+s1}{\PYZsq{}}\PY{l+s+s1}{.}\PY{l+s+s1}{\PYZsq{}}\PY{p}{)} \PY{o}{+} \PY{n+nb}{str}\PY{p}{(}\PY{n}{v}\PY{p}{)}\PY{o}{.}\PY{n}{rjust}\PY{p}{(}\PY{n}{right\PYZus{}width}\PY{p}{)}\PY{p}{)}
\end{Verbatim}


    \begin{Verbatim}[commandchars=\\\{\}]
{\color{incolor}In [{\color{incolor}18}]:} \PY{n}{breakfast\PYZus{}specials} \PY{o}{=} \PY{p}{\PYZob{}}\PY{l+s+s1}{\PYZsq{}}\PY{l+s+s1}{oats}\PY{l+s+s1}{\PYZsq{}}\PY{p}{:} \PY{l+m+mf}{3.95}\PY{p}{,} \PY{l+s+s1}{\PYZsq{}}\PY{l+s+s1}{banana}\PY{l+s+s1}{\PYZsq{}}\PY{p}{:} \PY{o}{.}\PY{l+m+mi}{95}\PY{p}{,} 
                               \PY{l+s+s1}{\PYZsq{}}\PY{l+s+s1}{eggs}\PY{l+s+s1}{\PYZsq{}}\PY{p}{:} \PY{l+m+mf}{2.50}\PY{p}{,} \PY{l+s+s1}{\PYZsq{}}\PY{l+s+s1}{milk}\PY{l+s+s1}{\PYZsq{}}\PY{p}{:} \PY{l+m+mf}{1.95}\PY{p}{,} \PY{l+s+s1}{\PYZsq{}}\PY{l+s+s1}{coffee}\PY{l+s+s1}{\PYZsq{}}\PY{p}{:}\PY{l+m+mf}{2.50}\PY{p}{\PYZcb{}}
\end{Verbatim}


    \begin{Verbatim}[commandchars=\\\{\}]
{\color{incolor}In [{\color{incolor}19}]:} \PY{n}{price\PYZus{}board}\PY{p}{(}\PY{n}{breakfast\PYZus{}specials}\PY{p}{,}\PY{l+m+mi}{40}\PY{p}{,}\PY{l+m+mi}{5}\PY{p}{)}
\end{Verbatim}


    \begin{Verbatim}[commandchars=\\\{\}]
==============Breakfast Specials=============
oats{\ldots} 3.95
banana{\ldots} 0.95
eggs{\ldots}  2.5
milk{\ldots} 1.95
coffee{\ldots}  2.5

    \end{Verbatim}

    \begin{Verbatim}[commandchars=\\\{\}]
{\color{incolor}In [{\color{incolor}20}]:} \PY{n}{price\PYZus{}board}\PY{p}{(}\PY{n}{breakfast\PYZus{}specials}\PY{p}{,}\PY{l+m+mi}{24}\PY{p}{,}\PY{l+m+mi}{10}\PY{p}{)}
\end{Verbatim}


    \begin{Verbatim}[commandchars=\\\{\}]
========Breakfast Specials========
oats{\ldots}      3.95
banana{\ldots}      0.95
eggs{\ldots}       2.5
milk{\ldots}      1.95
coffee{\ldots}       2.5

    \end{Verbatim}

    \hypertarget{rstrip-lstrip-strip}{%
\subsection{rstrip, lstrip, strip}\label{rstrip-lstrip-strip}}

\begin{itemize}
\tightlist
\item
  strip off whitespace

  \begin{itemize}
  \tightlist
  \item
    spaces
  \item
    tabs
  \end{itemize}
\end{itemize}

    \begin{Verbatim}[commandchars=\\\{\}]
{\color{incolor}In [{\color{incolor}96}]:} \PY{l+s+s1}{\PYZsq{}}\PY{l+s+s1}{    Hello World     }\PY{l+s+s1}{\PYZsq{}}\PY{o}{.}\PY{n}{strip}\PY{p}{(}\PY{p}{)} \PY{c+c1}{\PYZsh{} both left and right}
\end{Verbatim}


\begin{Verbatim}[commandchars=\\\{\}]
{\color{outcolor}Out[{\color{outcolor}96}]:} 'Hello World'
\end{Verbatim}
            
    \begin{Verbatim}[commandchars=\\\{\}]
{\color{incolor}In [{\color{incolor}97}]:} \PY{l+s+s1}{\PYZsq{}}\PY{l+s+s1}{    Hello World     }\PY{l+s+s1}{\PYZsq{}}\PY{o}{.}\PY{n}{lstrip}\PY{p}{(}\PY{p}{)} \PY{c+c1}{\PYZsh{} left}
\end{Verbatim}


\begin{Verbatim}[commandchars=\\\{\}]
{\color{outcolor}Out[{\color{outcolor}97}]:} 'Hello World     '
\end{Verbatim}
            
    \begin{Verbatim}[commandchars=\\\{\}]
{\color{incolor}In [{\color{incolor}98}]:} \PY{l+s+s1}{\PYZsq{}}\PY{l+s+s1}{    Hello World     }\PY{l+s+s1}{\PYZsq{}}\PY{o}{.}\PY{n}{rstrip}\PY{p}{(}\PY{p}{)} \PY{c+c1}{\PYZsh{} right}
\end{Verbatim}


\begin{Verbatim}[commandchars=\\\{\}]
{\color{outcolor}Out[{\color{outcolor}98}]:} '    Hello World'
\end{Verbatim}
            
    \begin{Verbatim}[commandchars=\\\{\}]
{\color{incolor}In [{\color{incolor}100}]:} \PY{l+s+s1}{\PYZsq{}}\PY{l+s+s1}{======Breakfast Specials======}\PY{l+s+s1}{\PYZsq{}}\PY{o}{.}\PY{n}{strip}\PY{p}{(}\PY{l+s+s1}{\PYZsq{}}\PY{l+s+s1}{=}\PY{l+s+s1}{\PYZsq{}}\PY{p}{)}
\end{Verbatim}


\begin{Verbatim}[commandchars=\\\{\}]
{\color{outcolor}Out[{\color{outcolor}100}]:} 'Breakfast Specials'
\end{Verbatim}
            
    \hypertarget{copy-and-paste-with-pyperclip}{%
\subsection{Copy and Paste with
Pyperclip}\label{copy-and-paste-with-pyperclip}}

\begin{itemize}
\tightlist
\item
  The pyperclip module has copy() and paste() functions that can send
  text to and receive text from your computer's clipboard.
\item
  Sending the output of your program to the clipboard will make it easy
  to paste it to an email, word processor, or some other software.
\end{itemize}

    \begin{Verbatim}[commandchars=\\\{\}]
{\color{incolor}In [{\color{incolor}21}]:} \PY{k+kn}{import} \PY{n+nn}{pyperclip}
         \PY{c+c1}{\PYZsh{}pyperclip.copy(\PYZsq{}Hello world!\PYZsq{})}
\end{Verbatim}


    \begin{Verbatim}[commandchars=\\\{\}]
{\color{incolor}In [{\color{incolor}23}]:} \PY{c+c1}{\PYZsh{} try ctrl + v}
\end{Verbatim}


    \begin{Verbatim}[commandchars=\\\{\}]
{\color{incolor}In [{\color{incolor}20}]:} \PY{n}{pyperclip}\PY{o}{.}\PY{n}{paste}\PY{p}{(}\PY{p}{)}
\end{Verbatim}


\begin{Verbatim}[commandchars=\\\{\}]
{\color{outcolor}Out[{\color{outcolor}20}]:} 'Hello world!'
\end{Verbatim}
            
    \begin{Verbatim}[commandchars=\\\{\}]
{\color{incolor}In [{\color{incolor}21}]:} \PY{c+c1}{\PYZsh{} Copy and paste anything}
\end{Verbatim}


    \begin{Verbatim}[commandchars=\\\{\}]
{\color{incolor}In [{\color{incolor}22}]:} \PY{n}{pyperclip}\PY{o}{.}\PY{n}{paste}\PY{p}{(}\PY{p}{)}
\end{Verbatim}


\begin{Verbatim}[commandchars=\\\{\}]
{\color{outcolor}Out[{\color{outcolor}22}]:} "In Automate the Boring Stuff with Python, you'll learn how to use Python to write programs that do in minutes what would take you hours to do by hand-no prior programming experience required. Once you've mastered the basics of programming, you'll create Python programs that effortlessly perform useful and impressive feats of automation to:"
\end{Verbatim}
            

    % Add a bibliography block to the postdoc
    
    
    
    \end{document}
